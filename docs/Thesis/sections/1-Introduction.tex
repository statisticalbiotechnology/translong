\chapter{Introduction}
\vspace{-0.75cm}
Gene regulation in mammalian cells is managed through a plethora of mechanisms, of which that of \acp{TF} is seen as one of the most important. These proteins indirectly regulate gene transcription through interactions with cis-regulatory elements (enhancers and promoters). These cis-regulatory element often have multiple 6-12 bp \ac{TF} binding sites, and  depending on the combination of \acp{TF} that bind the effect can be either repressing and activating. A \ac{TF} is thus not uniformly repressing or activating and is also not limited to regulating a single gene \cite{Spitz2012, lambert2018}. The complexity of the varied and cooperative interactions of \acp{TF} makes their effects difficult to understand and predict, but an important step in the pursuit of this goal is identifying their binding specificity, for which \ac{ChIP-seq} experiments has been one of the primarily methods \cite{lambert2018}. The work of investigating the binding specificity of the ~1500 known mammalian \acp{TF} using \ac{ChIP-seq} has benefited largely from the reduced cost of genome sequencing, but only about half have so far been investigated \cite{ISMARA2014}. Much of the work has been performed in larger projects such as The ENCODE Project (\url{https://www.encodeproject.org/}), but also several smaller projects have contributed. In an effort to make all this data comfortably available for data mining, a database called ChIP-Atlas was in 2015 introduced \cite{Oki2018}. Though ChIP-Altas is not the only database of its kind \cite{fornes2020jaspar,zheng2019cistrome, sloan2016encode, czipa2020chipsummitdb}, it is the most comprehensive and easily accessible for the purpose of this project.

The reduced cost of sequencing has meant a large influx of genetic data and the bottleneck in genetic research has thus largely been moved from laboratory experimentation to data analysis. This presents an opportunity for methods that can extract more information from transcript data, e.g. about the regulatory mechanisms controlling mRNA expression. Models for this purpose are commonly made where the mRNA expression levels of genes are calculated from the different levels of activity of their regulators \cite{Ouyanga2009,MARA2009,ISMARA2014}. From such model, the activity changes of regulators can be inferred from variations in mRNA expression data, which is what is done with the relatively well established tool MARA \cite{MARA2009}. It was in 2014 made into and online tool in the form of ISMARA \cite{ISMARA2014}. ISMARA bases its estimations on a limited set of promoters and their associated transcripts for which the binding sites of about 200 regulatory motifs has been predicted. When the user provides a set of mRNA expression data it is translated into a signal intensity for the different promoters. The data is put into two matrices, one describing the number of binding sites each motif has in each promoter and the other describing the signal associated with each promoter for each sample. This data is put into a model function that calculates the signal intensity from the number of binding sites, an unknown motif activity value and basal level of the promoters and samples. The motif activity is thus inferred from this model and presented with an activity profile across samples with an associated significance level. Besides the activity estimations, ISMARA further provides networks of known interactions, enriched Gene Ontology categories, etc. Though the method has seen success, it is due to the highly curated data used for analysis limited in which \acp{TF} can be assessed. Its model is also limited by the fact that it assumes TFs to either be primarily activating or repressing and thus handles balanced \acp{TF} poorly. 

MARA is however far from the only attempt at inferring the activity changes of regulators from the variations in expression of the genes which they regulate. Other examples include BASE \cite{Cheng2007}, RENATO \cite{Bleda2012}, VIPER \cite{Alvarez2016} and REACTIN \cite{Zhu2013}, as well as studies in which the method has not been named \cite{Schacht2014}. Many of these methods are network-based, which allows them to incorporate some of the complex cooperative interactions of \acp{TF}. Few are however focused on estimating and presenting activity changes occurring over time, which means that this is a niche that requires further research.

To further explore the possibilities of using mRNA expression data to accurately estimate how the activity of \acp{TF} change over time, a novel tool was in this project developed. The tool utilizes the factor analysis method \ac{PCA} to infer the activity of \acp{TF} and takes the approach of directly modeling the activity of a \ac{TF} from the mRNA expression of the genes it regulates. To assess the potential of the tool, it was applied to longitudinal mRNA expression data from \textit{Mus musculus} liver and brain, estimating the activities of 17 selected \acp{TF}.

\section{Problem}
\label{sec:problem}
Acquiring mRNA expression data is today simple, but experimentally measuring the activity of \acp{TF} remains tedious and costly. As the mRNA expression of a \ac{TF}'s gene is not an accurate representation of its activity, there is a need for methods that can utilize mRNA expression data to accurately estimate the activity of \acp{TF}. Current methods for this purpose are lacking, have issues such as being limited in terms of which \acp{TF} can be examined and not accommodating the fact that their regulatory effect can be equally activating and repressing. There is therefore a need to further explore alternative methods for \ac{TF} activity estimations.

\section{Goal}
\label{sec:goal}
This project aims to develop a simple, direct and widely applicable tool for estimating activities of \acp{TF} over time from longitudinal mRNA expression data, and in doing so test the potential of \ac{PCA} for inferring the activity of a \ac{TF} from the expression of the genes which it regulates.