\chapter{Discussion}
\vspace{-0.75cm}
\ac{PCA} is method that captures general trends in data, and the intent in this project was to isolate different trends by applying it to different extracts from a data set. This failed however, due to the strong common trend that was present in the data that caused all \ac{TF} activity estimations to be highly similar. This could be due to an inhered problem with the estimator and the assumptions it is based on, but could also be caused by problems with the data. If the trend was caused by a technical confounder the estimator could still have potential, but if the trend was indeed caused by biological factors it would mean that the assumptions were too distinct from reality. To determine the potential of the estimator further testing is therefore required.

The method applied in an effort to remove the confounder was, as the results showed, not fully successful. Removing the first \ac{PC} from the full data additionally meant that what was used for the analysis was no longer the actual measured mRNA expression, which complicated the interpretation of whatever results were obtained. It should also be considered that removing the first \ac{PC} from a data set means that a significant portion of the information the data contains is lost, and that it could potentially introduce new biases.

The estimator relies on the assumption that genes regulated by a common \ac{TF} share an identifiable unique common pattern in their mRNA expression, and that the \ac{TF} is the overall primary source of variation in their mRNA expression. This is however a gross simplification of the mechanics of \ac{TF} regulation. As described in Chapter \ref{sec:background}, many \acp{TF} act in a cooperative manner where the regulatory effect differs from the sum of its parts. Each gene regulated by a \ac{TF} has a different set of other regulators affecting its transcription, all with varying and changing levels of presence. The effect of the presences of the specific \ac{TF} could thus be either substantial, nonexistent or anything in between, depending on the gene and time point. It is thus unlikely that the investigated \ac{TF} is the primary source of variation in the mRNA expression of the genes in its set. Though, if the effects of the single \ac{TF} is not entirely obscured an idea could be to look into other \acp{PC} than the first to see if there are less obvious common pattern among the genes the \ac{TF} regulates. This however presents the issue of knowing which \ac{PC} to use for the estimation.

Another assumption of the model is that there is a linear correlation between \ac{TF} activity and mRNA expression of its gene. This does often not hold true in reality, even if all other prerequisites for its regulatory effect are fulfilled, as \acp{TF} can act closer to an on/off switch rather than increasing mRNA expression relative to concentration. Saturation effect could potentially also occur at high concentrations of a \ac{TF}, where there is no change in regulation despite increased concentration of the \ac{TF}. This assumption of a linear correlation is however inherited from the use of \ac{PCA} and must be accepted as a limitation of the estimator.

Without making major changes to the estimator, improved results could possibly be achieved by revising the gene sets used to define which genes individual \acp{TF} regulates. The sets directly taken from ChIP-Atlas are only a rough prediction of genes regulated by specific \acp{TF} based on proximity and are as a result also very large and likely to contain much noise. The method applied to compensate for this was, partly due to lack of options, equally crude and could with no certainty produce gene sets containing genes clearly regulated by the specified \ac{TF}. Though sets of genes with more assured association with each respective \ac{TF} are expected to become available in the future through the continued experimental efforts of mapping \ac{TF} regulation, it will take a long time until such data has the same coverage as ChIP-seq data has today. Another improvement that should be made is to make the gene sets for each organ individually since, as previously mentioned, \ac{TF} regulation in cases can be tissue specific.
