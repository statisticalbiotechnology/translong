\chapter{Future perspectives}
\vspace{-0.75cm}
From the tests performed in this project a final statement of the potential of the estimator can not be given for mainly two reasons; the expression data it was applied to and the method used for validation. 

To answer the question if the issues encountered in this project was caused by problems with the data or the estimator, further testing should be performed on other data sets. Parameters to consider for these include their complexity and the time frame of  sampling. In terms of complexity, the estimator could work better with data from simpler organisms, where there are less factors influencing mRNA expression. In terms of time frame, the time between sampling points was for the data used in this project multiple  days, so it could be of interest to test if the estimator can produce better results when estimating changed occurring over a shorter period. 

Comparing the estimated activity of a \ac{TF} to the mRNA expression of the gene coding for it, as done in this project, makes it impossible to claim that the estimator is any more accurate than simply looking at the mRNA expression. For further testing, other validation methods must therefore be employed. Ideal would be to compare estimated values to experimental values, which could be done after conducting a targeted experiment where, for a set period, the mRNA expression of all regulated genes is measured while in parallel performing an assay for measuring the activity of a selection of \ac{TF}. This way, estimations could be made with the mRNA expression data and the experimental activity values could be used as a ground truth for comparison. As such an experiment would be costly however, an alternative method could be to compare the results of the estimator to that of other available estimators, such as ISMARA. But without a reliable ground truth, it is difficult to make claims about the accuracy of the estimator.

 For further development on the estimator, the likely most important thing to look into is ways to take the cooperative binding of \acp{TF} into account. That this is necessary could not be determined in this project, but since it is key to their regulatory effects it can be expected to be significant for producing accurate estimations of \ac{TF} activity. 
 