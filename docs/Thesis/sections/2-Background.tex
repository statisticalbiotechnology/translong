\chapter{Background}
\label{sec:background}
\vspace{-0.75cm}
\section{Transcription factors}
\acp{TF} are proteins that indirectly regulate gene transcription through binding interactions with cis-regulatory elements commonly referred to as enhancers and promoters \cite{Spitz2012}, though the distinction between these is unclear and questioned \cite{Andersson2020}. A single \ac{TF} often has multiple binding sites of 6-12 more or less conserved bases \cite{lambert2018}, but the binding affinity is further heavily dependent on several other factors such as chromatin state, DNA methylation pattern \cite{Yineaaj2239} and the presence of other \acp{TF} \cite{jolma2015}. \acp{TF} rarely function on their own, requiring other \acp{TF} for opening up the chromatin and for various forms of cooperative binding \cite{lambert2018,Spitz2012}. The specifics of the interactions between the \acp{TF} and their binding sites is complex, variable and not fully understood, though three suggested general interaction models with supporting evidence exist; the enhanceosome model, the \ac{TF} collective model and the billboard model. These models suggest varying degrees of constrained binding, where a \ac{TF} can be more or less dependent on others for its binding and effect. It follows that the relation between a \ac{TF}'s concentration in a cell and the regulatory effect its presence has can vary greatly, eg. being linear, exponential or boolean in nature. Adding  to its complexity, the regulatory effect of a \ac{TF} can be either activating or repressing depending on which other \acp{TF} are present and to which regulatory element it binds, meaning that a \ac{TF} can generally not be categorized in such terms.

The actions of \acp{TF} has in cases been shown to be cell specific and key for spatiotemporal transcription regulation, and \acp{TF} are in general critically important during development \cite{Spitz2012, Gurdon2016}. The first step in understanding their roles and how they operate is to  find out where, when and under what conditions they bind. The binding specificity is commonly given as a "motif", a position weight matrix presented by a sequence logo where the occurrence of specific bases in each position is given. Determining these motifs has mostly been done through various \textit{in vitro} methods, but has also extensively been done \textit{in vivo} through \ac{ChIP} \cite{lambert2018}. The results of over 76000 \ac{ChIP-seq} experiments in \textit{Homo sapiens}, \textit{M. musculus}, \textit{Rattus norvegicus}, \textit{Drosophila melanogaster}, \textit{Caenorhabditis elegans} and \textit{Saccharomyces cerevisiae} reported in the Sequence Read Archives of NCBI (\url{https://www.ncbi.nlm.nih.gov/sra}) have been collected in a database dubbed ChIP-Atlas, from where lists of coding genes within 1, 5 or 10 kbp proximity of a \ac{TF}'s binding sites predicted using MACS2 peak calling can be retrieved for data-mining \cite{Oki2018}. This method of predicting which genes a \ac{TF} regulates has two major limitations; (1) it relies on the \acp{TF}' binding being functional and (2) it relies on the regulatory element only affecting nearby genes and not genes further away.

\section{Principal component analysis}
\acfi{PCA} is a statistical method applied to multivariate data to capture and condense variation. It does so by describing the data with a new set of orthogonal variables, the data's \acp{PC}, each constructed as a linear combinations of the original variables. The first \ac{PC} attempts to describe as much of the variation as possible and can be geometrically represented by a vector going through the center of the data. Its direction is such that the sum of all distances between data points and their projections onto the vector is minimized, which also maximizes the total distance between all projections onto the vector. Its magnitude is given by the amount of variance explained by the component, which can be compared to the total variance of the data for a measure of how much of the information in the data the \ac{PC} captures. \acp{PC} after the first have the same goal as the first, but with the additional constraint to be orthogonal to all prior \acp{PC} and will therefore attempt to describe as much as possible of the variation that has yet to be described by prior principal components \cite{Abdi2010}. 

The result of \ac{PCA} is thus a new set of variables that in descending order of importance encapsulates the most important information in the data. This is potent for dimensionality reduction, as a higher than proportional amount of the variance in the data can be represented with a reduced number of variables. If there are trends present in the data, these variables will also represent the most prominent trends, a feature that is utilized in this project. Another of the main uses of \ac{PCA} is for visualization of data with more than three variables, as such data can through \ac{PCA} potentially be accurately represented in a lower dimensional space where it can be easily plotted.

Within genetic research, very large data sets with high numbers of variables is very common. For example, many hundreds or even thousands of features are often compared between two or more conditions in efforts to identify possible causes of disease or phenotypic differences. As a relatively basic statistical method, \ac{PCA} is therefore commonly applied to such data to find important factors, as well as make it more interpretable and manageable \cite{Sevy2019, Ellsworth2017, Ji2013, Taguchi2018, Li2019}. In terms of research into \ac{TF} regulation, \ac{PCA} has previously been used for its ability to identify and extract trends and patterns \cite{Ouyanga2009}, though not in the same way it is applied in this project.
