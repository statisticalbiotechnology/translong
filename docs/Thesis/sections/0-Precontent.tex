% Frontmatter includes the titlepage, abstracts and table-of-contents
\frontmatter

\titlepage

\begin{abstract}
  \textit{Transcription factors} (TFs) are key regulatory proteins that regulate transcription through precise, but highly variable binding events to cis-regulatory elements. The complexity of their regulatory patterns makes it difficult to determine the roles of different TFs, a task which the field is still struggling with. Experimental procedures for this purpose, such as knock out experiments, are however costly and time consuming, and with the ever-increasing availability of sequencing data, computational methods for inferring the activity of TFs from such data have become of great interest. Current methods are however lacking in several regards, which necessitates further exploration of alternatives.
  
  A novel tool for estimating the activity of individual TFs over time from longitudinal mRNA expression data was in this project therefore put together and tested on data from \textit{Mus musculus} liver and brain. The tool is based on principal component analysis, which is applied to data subsets containing the expression data of genes likely regulated by a specific TF to acquire an estimation of its activity. Though initial tests on 17 selected TFs showed issues with unspecific trends in the estimations, further testing is required for a statement on the potential of the estimator.
  
  \subsection*{Keywords}
  Factor analysis, principal component analysis, transcription factors, transcriptomics, ChIP, \textit{Mus musculus}
\end{abstract}

\begin{otherlanguage}{swedish}
  \begin{abstract}
    \textit{Transcriptionsfaktorer} (TFer) är viktiga regulatoriska protein som reglerar transkription genom att binda till cis-regulatoriska element på precisa, men mycket varierande vis. Komplexiteten i deras regulatoriska mönster gör det svårt att avgöra vilka roller olika TFer har, vilket är en uppgift som fältet fortfarande brottas med. Experimentella procedurer i detta syfte, till exempel "knock out" experiment, är dock kostsamma och tidskrävande, och med den evigt ökande tillgången på sekvenseringsdata har metoder för att beräkna TFers aktivitet från sådan data fått stort intresse. De beräkningsmetoder som finns idag brister dock på flera  punker, vilket erfordrar ett fortsatt sökande efter alternativ.
    
    Ett nytt vektyg för att upskatta aktiviteten hos individuella TFer över tid med hjälp av longitunell mRNA-uttrycksdata utvecklades därför i det här projektet och testades på data från \textit{Mus musculus} lever och hjärna. Verktyget är baserat på principalkomponentsanalys, som applicerades på set med uttrycksdata från gener sannolikt reglerade av en specifik TF för att erhålla en uppskattning av dess aktivitet. Trots att de första testerna för 17 utvalda TFer påvisade problem med ospecifika trender i upskattningarna krävs forsatta tester för att kunna ge ett tydligt svar på vilken potential estimatorn har.
    \subsection*{Nyckelord}
    Faktoranalys, principalkomponentanalys, transkriptionsfaktorer, traskriptomik, ChIP, \textit{Mus musculus}
  \end{abstract}
\end{otherlanguage}



  