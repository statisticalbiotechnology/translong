\chapter{Materials and Methods}
\vspace{-0.75cm}

The estimator was entirely programmed in Python version 3.7.4, using the Pandas \cite{mckinney-proc-scipy-2010} and NumPy \cite{van2011numpy, oliphant2006guide} packages for data management and NumPy and the NumPy-based package scikit-learn \cite{scikit-learn} for data analysis. For additional data analysis, the NumPy-based package SciPy \cite{SciPy}  was additionally used. All code written for this project is available at \url{https://github.com/statisticalbiotechnology/translong/tree/ReportVersion}.

\section{Estimator algorithm}

 The algorithm used by the estimator for calculating the activity of a \ac{TF} includes the following main steps, each further outlined below:
\begin{enumerate}
    \item Prepare a data set including singular values of the transcript levels of different genes in different samples
    \item Create a list of genes with indications of being regulated by the \ac{TF}
    \item From the prepared data set, extract the subset of the transcript levels of the genes included in the \ac{TF}'s gene list
    \item Perform \ac{PCA} on the extracted subset of data to acquire a variable that represents the activity of the \ac{TF}
\end{enumerate}

\noindent The estimator was made to handle longitudinal mRNA count data where reads have been mapped to a reference genome, thus in the format of a list of genes that each has single values of their transcript level in each sample. The data is as mentioned managed using the Pandas package, with which it is loaded into a data frame whose row index specified the gene and column index specified the sample. Note that replicate samples are  handled no differently than samples from different conditions. The data is then preprocessed through sample-wise \ac{RPM} normalization, removal of genes with no measured expression in any sample and log2-transformation. \ac{RPM} normalization is performed as a standard method of normalizing count data. Genes with no expression are removed to enable log2-transformation as well as to avoid biases in the data caused by no expression being measured for a number of genes in a sample due to potential measurement errors. Log2-transformation is performed as count data is commonly assumed to be log-normally distributed and the \ac{PCA} later applied works under the assumption that the data is normally distributed.

For the \acp{TF} to be analysed, the estimator requires information of which genes each \ac{TF} is believed to regulate. This information is stored in a second data frame, where the row-index specifies the \ac{TF} and the first column of each rows contain the list of all genes believed to be regulated by the \ac{TF}. For each \ac{TF} in the index, transcript levels of the genes in their gene list is picked out from the first data frame, resulting in \ac{TF} specific subsets of transcription data. Each subset is then sample-wise standardized to be centered to its mean and scaled to unit variance.

\ac{PCA} is finally applied to each subset of standardized transcription data using the \ac{PCA} function from the scikit-learn package, calculating the first \ac{PC} of the subsets. The results for each \ac{TF} is thus a single value for each sample which, given the assumption that the \ac{TF} is the primary source of variation in the subset of data, describes the activity of the \ac{TF} at that sampling point. Additionally, the percentage of total variance in the subsets that is explained by each \ac{PC} and the contribution each gene in the set have to each \ac{PC} is given along with the activity estimation.
 
\subsection{Algorithm options}

An alternative to removing all genes with no measured expression in any sample was implemented in the estimator, where only rows for which more than 25\% of their values are 0 are removed. To enable log2-transformation, a value of 1 is then added to all remaining values of the data set. As this method is less strict, it preserves more data, but comes at the cost of requiring modifications of the data. Furthermore, to prevent missing measurements from affecting the results, a weighed \ac{PCA} \cite{wPCA} is then used where the weights of 0-values is set to 0 and the rest to 1.

Though the first \ac{PC} was intended to be the estimation of the \ac{TF} activity, the option to extract any number of components for each subset of transcription data is given. Note however that the number of components can not exceed the number of original variables, and can thus not be higher than the number of samples or the number of genes included in the \ac{TF}'s gene list.

Due to a confounder being found during analysis, a method for cleaning the data from possible linear confounders was implemented. The method utilizes \ac{PCA} and consists of the option to remove one or more of the fist \acp{PC} from the full set of transcription data prior to further analysis. This is done by performing \ac{SVD} on the data and reconstructing it using all except the \textit{N} first \acp{PC}, where \textit{N} is a variable given when the estimator function is called.

\section{Estimator testing}

\subsection{Longitudinal mRNA expression data}

The raw data analysed in this project to test the estimator was kindly provided by Claudia Kutter's lab at SciLifeLabs and is publicly available from the article from which it originates \cite{Schmitt2014}. It consisted of quantitative mRNA expression count data from \textit{M. musculus} brain and liver, where expression levels of the genes included in Ensenbl release 57 gene annotation of \textit{M. musculus} \cite{Flicek2014} had been assigned from the measured reads using HTSeq-count \cite{Anders2014}. It included quantitative mRNA expression levels for 37991 genes from samples taken from brains and livers at six different time points during the mice's early development: 15.5 and 18.5 days into embryonic development and 0.5, 4, 22 and 29 days  into postnatal development. For each sampling point, two biological replicates had been taken in both brain and liver, adding up to a total of 24 samples. Two technical replicates had also been taken and combined for each sample. 

As reads had been assigned to an Ensembl reference genome, genes were in this data labeled with Ensembl IDs. For the analysis these had to be converted into gene symbols. This was done via Biomart \cite{Smedley2015}, from where an unfiltered list of Ensembl IDs and their associated gene symbols was extracted for \textit{M. musculus} build GRCm38.p6 using the standard options. Genes with no gene symbol or with more than one gene symbol in the list were discarded. The conversion resulted in the number of genes being reduced to 32394.

The sample behaviour was examined by plotting the full data's first \ac{PC} against its second \ac{PC}, which showed that the first principal component of the full mRNA expression data separated samples by which organ they were sampled in (see Figure \ref{fig:SampleBehaviour} in the appendix). In order for the estimator to capture specific patterns of the \acp{TF} rather than general trends that separated the organs, the data from brain and liver was thus analysed separately for the estimations. This was further motivated by the fact that the activity of transcription factors has been shown to be cell type specific \cite{Gurdon2016}.

\subsection{Transcription factor targets}

Lists of genes believed regulated by each \ac{TF} were defined using the \textit{M. musculus} data from the ChIP-Atlas database \cite{Oki2018}. Genes are there categorized as potentially regulated by a specific \ac{TF} if their transcription start site is located within a 10 kbp proximity of one of the \ac{TF}'s experimentally found binding sites. Gene lists for a total of 699 \acp{TF} were downloaded, each including the gene symbols of the genes categorized as regulated by the \ac{TF} and probability scores, called MACS2 scores, given by the peak calling method for each gene reported in each experiment. 

Proximity to a \ac{TF}'s binding site is however only loose evidence for interactions, so the gene lists were expected to contain high levels of noise in the form of genes with no actual association to the \ac{TF}. This was to a larger degree true for gene lists with a high number of genes, which was often the case for \acp{TF} that had been investigated in a large number of experiments, and given that the size of gene lists varied from including just a few genes to multiple thousands, there was a need to reduced their size. The gene lists were therefore filtered on two properties; minimum MACS2 score and maximum number of genes. For each gene associated with a \ac{TF}, the mean MACS2 scores were calculated for data reported from experiments performed on samples of similar origin. For a gene to be included in a gene list at least one of these means had to be above the arbitrarily selected value of 300. If more than the maximum number of genes fulfilled this criteria, the 100 genes with the highest mean MACS2 scores were selected.

\subsection{Activity analysis}

The estimator was first applied to 17 selected \acp{TF}, estimating their activity in both liver and brain using the standard options; removing all genes with a transcription level of 0 in any sample, using the first \ac{PC} of each \ac{TF}'s data subset for their activity estimations, and without applying the method for cleaning the data from linear confounders. As a form of validation, activity estimations of each analysed \ac{TF} was compared to the mRNA expression of their own genes in line plots to see if they shared common patterns. This was done on the basis of there being a correlation between gene expression and protein activity, which has been shown to be true to varying degrees \cite{Liu2016}. The correlation was thus expected to be far from perfect, but the method was employed due to a lack of alternatives. 

The selection of the 17 \acp{TF} to be analysed had been done by picking out the 10 \acp{TF} that in the \textit{M. musculus} data for each organ had the highest variance among those with above mean mRNA expression. These were selected as they were deemed more likely to be a large influence on the variation in mRNA expression of the genes they regulate. They were also likely to have distinguishable mRNA expression patterns that could be compared to their activity estimations. The \ac{TF} selected from brain data were Sox11, Zfp57, Top2a, Klf9, Lmnb1, Sox4, Satb2, Tcf7l2, Nr1d2 and Nr3c1. The \ac{TF} selected from liver data were Tal1, Lmo2, Ncapg, Ikzf1, Top2a, Klf9, Uhrf1, Brca1, Mllt3 and Rb1. Uhrf1 was however disregarded due to the limited number of genes included in its gene set.

After observing the effects of a possible covariate in the mRNA expression data, the activities of the 17 \acp{TF} were estimated in both organs a second time. The optional data-cleaning method was this time applied to remove the first \ac{PC} from the full normalized data of each organ before the estimations. The results were again plotted in line plots, using same validation method was previously.
